\begin{thebibliography}{99}
%% La bibliografía se ordena en orden alfabético respecto al apellido del 
%% autor o autor principal
%% cada entrada tiene su formatado dependiendo si es libro, artículo,
%% tesis, contenido en la web, etc

% artículo en una revista arbitrada
\bibitem{albin} P. Albin, E. Leichtnam, R. Mazzeo y P. Piazza. The signature package on Witt spaces. \textit{Ann. Sci. Ec. Norm. Supér. (4)}, \textbf{45}(2):241--310, 2012.

% libro
\bibitem{Brez} H. Brezis. \textit{Analyse functionnelle, théorie et applications.} (Collection Mathématiques Appliquées pour la Maítrise) Masson, Paris, 1992.

% libro
\bibitem{Choq} Y. Choquet"=Bruhat y otros. \textit{Analysis, manifolds and physics. (volumen 1)} North"=Holland Publishing Company, Amsterdam, 1996.

% libro
\bibitem{C:H2} R. Courant y {D. Hilbert}. \textit{Methods of mathematical physics. (volumen 2)} Interscience Publishers, Nueva York, 1962.

% artículo en la web arXiv, preprint
\bibitem{Rafa4} R. {De la Madrid}. The rigged {Hilbert} space of the free hamiltonian. Consultado en marzo de 2005 en \url{http://arxiv.org/abs/quant-ph/0210167}.

% libro con datos faltantes, s.e. "sin editorial"
\bibitem{Doc} J. Escamilla-Castillo. \textit{Topología.} 2"a ed. s.e., Guatemala, 1992.

% libro traducido
\bibitem{Hasr} N. Haaser y {J. Sullivan}. \textit{Análisis real.}  Tr.~Ricardo Vinós. Trillas, México, 1978.

% libro traducido
\bibitem{Halmo} P. Halmos. \textit{Teoría intuitiva de los conjuntos.} 8"a ed. Tr.~Antonio Martín. Compañía Editorial Continental, S.A., México, 1973.

% libro 
\bibitem{Haus} F. Hausdorff. \textit{Set theory.} 2"a ed. Chelsea Publishing Company, Nueva York, 1962.

% libro
\bibitem{Heis} W. Heisenberg. \textit{The physical principles of the quantum theory.} Dover Publications, Inc., Nueva York, 1949.

% libro
\bibitem{Hewit} E. Hewitt y {K. Stromberg}. \textit{Real and abstract analysis.} Springer"=Verlag, Nueva York, 1965.

% libro
\bibitem{Kolmo} A. Kolmogorov y {S. Fomin}. \textit{Elementos de la teoría de funciones y del análisis funcional.} Tr.~Carlos Vega. MIR, Moscú, 1975.

% artículo de una enciclopedia en línea
\bibitem{Kronz} F. Kronz. Quantum theory: von {Neumann} versus {Dirac}. Consultado en marzo de 2005 en \url{http://plato.stanford.edu/entries/qt-nvd/}.

% artículo en una revista arbitrada
\bibitem{liu} K. Liu, X. Sun, and S.-T. Yau. Goodness of canonical metrics on the moduli space of Riemann surfaces. \textit{Pure Appl. Math. Q.}, \textbf{10}(2):223--243, 2014

% artículo en una revista arbitrada
\bibitem{leader} E. Leader and C. Lorcé, The angular momentum controversy: What's it all about and does it matter?, \textit{Phys. Rept.} \textbf{541}, 163 (2014).

% libro
\bibitem{Omnes} Omn\`{e}s, R. \textit{The interpretation of quantum mechanics.} (Princeton Series in Physics) Princeton University Press, Princeton, 1994.

% libro
\bibitem{Penr} R. Penrose. \textit{La mente nueva del emperador.} Tr.~José García. Fondo de Cultura Económica, México, 1996.

% publicación electrónica
\bibitem{Ster} S. Sternberg. Theory of functions of a real variable. Consultado en abril de 2005 en \url{http://www.math.harvard.edu/\~shlomo}.

% publicación electrónica
\bibitem{Tesc} G. Teschl. Mathematical methods in quantum mechanics with applications to {Schrödinger} operators. Consultado en abril de 2005 en \url{http://www.mat.univie.ac.at/\~gerald}.

\end{thebibliography}
