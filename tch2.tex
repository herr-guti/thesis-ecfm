\chapter{CITAS BIBLIOGRÁFICAS}

\begin{thm}
	Un teorema para este capítulo.
\end{thm}

\begin{center}
	\begin{tabular}{c|c}
		\hline
		Comando	& Salida  \\
		\hline
		\verb|\cite{Brez}| & \cite{Brez} \\
		\hline
		\verb|\cite[sufijo]{Hewit}| & \cite[sufijo]{Hewit} \\
		\hline
		\verb|\cite[prefijo][sufijo]{Brez}| & \cite[prefijo][sufijo]{Brez} \\
		\hline
		\verb|\citet{Brez,liu}| & \citet{Brez,liu} \\
		\hline
		\verb|\citep{Brez,liu}| & \citep{Brez,liu} \\
		\hline
		\verb|\citealt{Brez,liu}| & \citealt{Brez,liu} \\
		\hline
		\verb|\citealp{Brez,liu}| & \citealp{Brez,liu} \\
		\hline
		\verb|\citeauthor{liu}| & \citeauthor{liu} \\
		\hline
		\verb|\citeauthor*{liu}| & \citeauthor*{liu} \\
		\hline
		\verb|\citeyear{Brez,Hewit}| & \citeyear{Brez,Hewit} \\
		\hline
		\verb|\citeyearpar{Brez,Hewit}| & \citeyearpar{Brez,Hewit} \\
		\hline
	\end{tabular}
\end{center} Los comando aceptan \verb|[sufijo]| y \verb|[prefijo]|, el segundo requiere agregar un sufijo.

En algunos casos es necesario dar la cita completa (es costumbre en algunos ámbitos al citar por primera vez), el código \verb|\citefull{liu}, p. 573| produce \citefull{liu}, p. 573.